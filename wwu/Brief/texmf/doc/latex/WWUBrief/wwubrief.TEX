% Dies ist WWUBRIEF.TEX
\def\Datum{7. Mai 2008}
% Änderungen:
% 07.05.2008 Kr     Angepasst an WWU-CD und umbenannt in wwubrief
% 25.02.2002 Kr     inputenc utf8
% 15.12.1995 Kr     Umstellung auf LaTeX2e
% 03.09.1992 Ar/FP  Ausgangsversion
% --------------------------------------------------------------------------
\documentclass{wwubrief}
\usepackage{ngerman}
\usepackage[utf8]{inputenc}
\usepackage{alltt,expdlist}
\renewcommand{\rmdefault}{cmr}

\def\ttbackslash{{\tt\symbol{92}}}        % typewriter \
\def\ttlbrace{{\tt\symbol{123}}}          % typewriter {
\def\ttrbrace{{\tt\symbol{125}}}          % typewriter }
\long\def\specialitem#1%
       {\vspace{4ex}\par%
        \item[{#1}]%
        \rule{0mm}{0mm}\hss\\[2ex]}


\begin{document}
\begin{center}
\thispagestyle{empty}
\vspace*{10ex}
{\huge\bf Erstellen von Briefen mit Universitätsbriefkopf \par}
\vspace*{\fill}
\vspace*{10ex}

{\Large Westfälische Wilhelms-Universität Münster\\
        Zentrum für Informationsverarbeitung\\[1ex]
         {St. Arntzen, W. Kaspar, F. Perske}\par}
\vspace*{\fill}
\vspace*{\fill}
{\Large \Datum}
\vspace*{30ex}
\end{center}
\newpage

% Deckelrückseite leer
\mbox{}
\thispagestyle{empty}
\newpage

\setcounter{page}{1}

\parskip 2mm                            % 3.9.92 FP
\absendername{Stefan Arntzen\\Wolfgang Kaspar}
\telefon{24\,73}
\begin{center}
\vskip2em
{\LARGE Erstellen von Briefen mit
            Universitätsbriefkopf}
\vskip1.5em
{\large Westfälische Wilhelms-Universität Münster\\
        Zentrum für Informationsverarbeitung\\
         {\normalsize St. Arntzen, W. Kaspar, F. Perske}}
\vskip1em
{\large \Datum}
\vskip1.5em
\end{center}

Über die vom Zentrum für Informationsverarbeitung (ZIV) entwickelte
\LaTeX"=Dokumentenklasse {\tt BRIEF.CLS} werden \LaTeX-Markierungen
zur Verfügung gestellt, die das Erstellen von Briefen im neuen
Corporate Design der WWU unterstützen.

In der vorliegenden Anleitung wird vorausgesetzt, daß dem Anwender das
Arbeiten mit \LaTeX\ vertraut ist.

{\bf Aufbau eines \LaTeX-Briefes}

Die Eingabedatei eines \LaTeX-Briefes ist folgendermaßen gegliedert:
%
\begin{quotation}\noindent
\verb+\documentclass{wwubrief}+\\
\verb+\usepackage{ngerman}+\\
\verb+\usepackage[utf8]{inputenc}+
\begin{verbatim}
\begin{document}\end{verbatim}
\verb+\begin{brief}{+{\it
Briefkopfdatei\/}\verb+}+
\vspace{.2cm}

\noindent\rule{0mm}{0mm}\verb+       +{\it Briefvorspann\/}
\vspace{.2cm}

\noindent\verb+\text+
\vspace{.2cm}

\noindent\rule{0mm}{0mm}\verb+       +{\it Briefrumpf\/}
\vspace{.2cm}

\noindent\verb+\end{brief}+\\
\verb+\end{document}+
\end{quotation}

Als Dokumentenklasse in der \verb+\documentclass+-Markierung muß {\tt wwubrief}
angegeben werden.
Diese Klasse lädt zusätzlich noch das \texttt{graficx}-Paket
und setzt das Fontencoding auf \texttt{T1}.

Der \LaTeX-Brief wird durch die Markierung \verb+\text+ in zwei
Abschnitte unterteilt.
Der erste Abschnitt ist der Briefvorspann. Er umfaßt den Brief bis
einschließlich der Anrede.
Der zweite Abschnitt -- der Briefrumpf -- enthält den Brieftext,
gefolgt vom Gruß, von Hinweisen auf Anlagen etc.

Der Briefvorspann beginnt mit der Markierung

\begin{quotation}\noindent
\verb+\begin{brief}{+{\it Briefkopfdatei\/}\verb+}+
\end{quotation}

\pagebreak[4]
Im Vorspann dürfen folgende Markierungen benutzt werden:
\vspace*{-4ex}
\begin{quotation}\noindent
\begin{description}[\setlabelsize{64mm}]
\item[{\tt \string\kopien\ttlbrace{\it Nummer\/}\ttrbrace}]
        Anzahl von Kopien des Briefes
\item[{\tt \string\absendername\ttlbrace{\it Name\/}\ttrbrace}]
        Name des Absenders
\item[{\tt \string\telefon\ttlbrace{\it Telefonnummer\/}\ttrbrace}]
        Telefonnummer des Absenders
\item[{\tt \string\fax\ttlbrace{\it Nummer\/}\ttrbrace}]
        Fax-Nummer des Absenders
\item[{\tt \string\userid\ttlbrace{\it Benutzerkennung\/}\ttrbrace}]
        Benutzerkennung des Absenders
\item[{\tt \string\node\ttlbrace{\it Rechneradresse\/}\ttrbrace}]
        Rechneradresse für Benutzerkennung
\item[{\tt \string\datum\ttlbrace{\it Datum\/}\ttrbrace}]
        Datum für den Briefkopf
\item[{\tt \string\az\ttlbrace{\it Aktenzeichen\/}\ttrbrace}]
        Aktenzeichen
\item[{\tt \string\adresse\ttlbrace{\it Empfängeradresse\/}\ttrbrace}]
        Adresse des Empfängers
\item[{\tt \string\mehrfachadresse\ttlbrace{\it Empfängeradressen\/}\ttrbrace}]
        Mehrere Empfängeradressen auf einem Brief
\item[{\tt \string\adressdatei\ttlbrace{\it Dateiname\/}\ttrbrace}]
        Gleicher Brief an mehrere Adressaten. Einzelne Adressen
        stehen in einer Datei.
\item[{\tt \string\bezugszeichen\ttlbrace{\it Textzeile\/}\ttrbrace}]
        Bezugszeichen
\item[{\tt \string\betr\ttlbrace{\it Textzeile\/}\ttrbrace}]
        Betreff
\item[{\tt \string\bezug\ttlbrace{\it Textzeile\/}\ttrbrace}]
        Bezug
\item[{\tt \string\anrede\ttlbrace{\it Textzeile\/}\ttrbrace}]
        Anrede des Empfängers
\item[{\tt \string\grussformel\ttlbrace{\it Textzeile\/}\ttrbrace}]
        Grußformel, die am Ende des Briefes gedruckt werden soll.
\end{description}
\end{quotation}

Die Argumente der Markierungen des Briefvorspanns werden nicht
sofort auf der Seite ausgegeben sondern erst zwischengespeichert. Erst
beim Abschluß des Briefvorspanns mittels der Markierung
\verb+\text+
werden alle diese Argumente zusammen mit den restlichen Angaben des
Briefkopfes ausgegeben. Deshalb kann im Briefvorspann die Reihenfolge
der Markierungen beliebig gewählt werden.

Die Markierung

\begin{quotation}\noindent
\verb+\text+
\end{quotation}

schließt den Vorspann ab und eröffnet den Rumpf des Briefes,
in dem folgende Markierungen zur Verfügung stehen:
\vspace*{-4ex}
\begin{quotation}\noindent
\begin{description}[\setlabelsize{64mm}]
\item[{\tt \string\schluss}]
        Rest wird zusammenhängend gedruckt
\item[{\tt \string\gruss\ttlbrace{\it Unterzeichner\/}\ttrbrace}]
        Gruß mit Unterschrift
\item[{\tt \string\anlage\ttlbrace{\it Anlagenliste\/}\ttrbrace}]
        Auf Anlagen verweisen
\item[{\tt \string\ps\ttlbrace{\it Nachsatz\/}\ttrbrace}]
        Postskriptum
\end{description}
\end{quotation}

Im Gegensatz zum Vorspann lösen die Markierungen im Rumpf des Briefes
direkt die Ausgabe der jeweiligen Argumente aus. Die
Reihenfolge dieser Markierungen entscheidet also darüber, in welcher
Reihenfolge die jeweiligen Ausgaben gedruckt werden sollen.

\pagebreak[4]
Die Markierung

\begin{quotation}\noindent
\verb+\end{brief}+
\end{quotation}

schließt den Rumpf und damit den gesamten Brief ab.
\vspace{.5cm}

{\bf Beschreibung der \LaTeX-Markierungen}

Um den Corporate-Design-Vorgaben der WWU
zu genügen, sollten die im Folgenden aufgeführten  Stilparameter
möglichst nicht geändert werden
(siehe auch http://www.uni-muenster.de/corporate-design/).
%
\begin{description}[\setlabelsize{15mm}] % 3.9.92 FP
\vspace*{-4ex}
\specialitem{\tt \ttbackslash begin\ttlbrace brief\ttrbrace%
                              [{\it Bearbeiterdatei\/}]%
                              \ttlbrace{\it Briefkopfdatei\/}\ttrbrace%
                              [{\it Stilparameter\/}]%
            }
Markierung des Briefbeginns. {\it Briefkopfdatei\/} ist
dabei der Name einer {\tt STY}-Datei, die das Aussehen des Briefkopfes
festlegt. Der Briefkopf des Universitätsrechenzentrums ist z.B. in der
{\it Briefkopfdatei\/} \verb+URZKOPF.STY+ definiert.
Die {\tt STY}-Datei {\it Bearbeiterdatei\/} enthält
Markierungen, die in allen Briefen eines Bearbeiters immer den gleichen
Parameter besitzen (z.B. Name, Telefonnummer).
Die zulässigen {\it Stilparameter\/} sind in der folgenden Liste
beschrieben. {\it n\/} steht für eine ganze Zahl ohne Maßeinheit.
Sie gibt, wenn nichts anderes beim jeweiligen Stilparameter gesagt wird,
den Abstand in Millimetern zum vorhergehenden Briefteil an. Negative
oder gebrochene Zahlen sind erlaubt (z.B.~-6.9).


\begin{description}[\setlabelsize{10mm}] % 28.7.92 FP
\item[{\bf Stilparameter}]{\bf Beschreibung}

\item[{\tt \string\absname\ttlbrace an/aus/{\it n\/}\ttrbrace}]
       Festlegen
       ob, bzw. wo der Absendername gedruckt werden soll.\\
       Voreingestellt ist: \verb+\absname{2}+.

\item[{\tt\string\absanschrift\ttlbrace an/aus/{\it n\/}\ttrbrace}]
       Festlegen, ob die Absenderanschrift -- Straße und Ort -- im
       rechten Teil des Briefkopfes gedruckt werden sollen.\\
       Voreingestellt ist: \verb+\absanschrift{1}+.

\item[{\tt\string\absimfenster\ttlbrace an/aus\ttrbrace}]
       Die Absenderzeile im Brief"|fenster soll, bzw. soll nicht
       gedruckt werden.\\
       Voreingestellt ist: \verb+\absimfenster{an}+.

\item[{\tt\string\dat\ttlbrace an/aus/{\it n\/}\ttrbrace}]
       Festlegen ob, bzw. wo das Datum gedruckt werden soll.\\
       Voreingestellt ist: \verb+\dat{1}+.

\item[{\tt\string\kopfsep\ttlbrace{\it n\/}\ttrbrace}]
       Angeben, wo der Brieftext einschließlich Betreff und
       Bezug anfangen soll. Es wird {\it n\/}
       Millimeter unter dem Briefkopf begonnen. Es wird jedoch immer
       soweit nach unten gerückt, daß der Text nicht im
       Brief"|fenster anfängt.\\
       Voreingestellt ist: \verb+\kopfsep{0}+.

\ifx\DeclareFontShape\undefined  % LaTeX 2.09
\else  %  nur für LaTeX2e
%\enlargethispage{2ex}
\fi
\item[{\tt\string\num\ttlbrace an/aus/{\it n\/}\ttrbrace}]
       Seitennumerierung ein-/ausschalten, bzw. bei der Seitenzahl
       {\it n\/} starten. Wenn die Numerierung eingeschaltet ist, werden
       die Seitennummern erst ab der zweiten Seite ausgedruckt!
       Entsprechend erhält die zweite Seite des Briefes den Wert $n+1$,
       wenn der Wert $n$ angegeben wurde.\\
       Voreingestellt ist: \verb+\num{an}+.

\pagebreak[4]
\item[{\tt\string\randoben\ttlbrace{\it n\/}\ttrbrace}]
       Explizite Angabe des oberen Seitenrandes. Gilt nur für die
       zweite und die folgenden Seiten, da sich oben auf der ersten
       Seite der Briefkopf befindet. Analog zum Stilparameter
       \verb+\randunten+ ist zu berücksichtigen, daß zwischen oberem
       Rand und Text eine Seitenkopfzeile angeordnet ist.\\
       Voreingestellt ist: \verb+\randoben{20}+.

\item[{\tt\string\randunten\ttlbrace{\it n\/}\ttrbrace}]
       Explizite Angabe des unteren Seitenrandes.
       Bei Angabe dieses Stilparameters ist zu beachten, daß sich
       zwischen Text und unterem Rand noch die Seitenfußzeilen
       befinden.\\
       Voreingestellt ist: \verb+\randunten{24}+.

\item[{\tt\string\faltmarken\ttlbrace an/aus\ttrbrace}]     %
       Die Faltmarken am linken Rand der ersten Seite       %
       sollen, bzw. sollen nicht gedruckt werden.\\         %
       Voreingestellt ist: \verb+\faltmarken{an}+.          % ^ 28.7.92 FP
\end{description}


\specialitem{\tt \ttbackslash kopien\ttlbrace {\it Anzahl\/}\ttrbrace}
Die Anzahl der Kopien, die von dem Brief gedruckt werden sollen, kann
hier angegeben werden. Voreingestellt ist: \verb+\kopien{0}+. Oben
Rechts auf der jeweils ersten Seite einer jeden Kopie
kann ein Ablagevermerk ausgegeben werden. Standartmäßig ist das
{\it 1. Kopie\/}, {\it 2. Kopie\/} usw. Genaueres dazu im Kapitel
\frqq Zusätzliche Parameter\flqq.

\specialitem{\tt \ttbackslash absendername\ttlbrace{\it Name\/}\ttrbrace}
Über diese Markierung wird der Name des Bearbeiters angegeben.
\begin{quotation}\noindent
\verb+\absendername{Hans Maier}   ==>  +
{\scriptsize Bearbeiter\ \ \ \ Hans Maier}
\end{quotation}

\specialitem{\tt \ttbackslash bearbeitertext\ttlbrace{\it Bezeichnung\/}\ttrbrace}
Als \textit{Bezeichnung} kann z.\,B.
\frqq Bearbeiterin\flqq\ eingetragen werden.
\begin{quotation}\noindent
\parbox[t]{65mm}%
{\ttbackslash bearbeitertext\ttlbrace Bearbeiterin\ttrbrace\hfill\texttt{==>\ \ }\\
 \ttbackslash absendername\ttlbrace Martina Mustermann\ttrbrace}
{\scriptsize Bearbeiterin\ \ \ Martina Mustermann}
\end{quotation}

\specialitem{\tt \ttbackslash telefon%
                              \ttlbrace{\it Telefonnummer\/}\ttrbrace}
Die Telefonnummer des Absenders wird in die Liste der Telefonnummern
eingefügt.
\begin{quotation}\noindent
\verb+\telefon{31673}            ==>  +%
{\scriptsize Tel. +49\,251\,83-31673}
\end{quotation}

\specialitem{\tt \ttbackslash userid%
                     [\ttbackslash node%
                           \ttlbrace{\it Rechneradresse\/}\ttrbrace]%
                     \ttlbrace{\it Benutzerkennung\/}\ttrbrace%
            }
Die Benutzerkennung des Absenders wird als E-mail-Adresse eingefügt.
Als optionales Argument kann noch die Markierung \verb+\node+
angegeben werden.
\begin{quotation}\noindent
\verb+\userid{kaspar}              ==>  +%
{\scriptsize kaspar@uni-muenster.de}
\end{quotation}

\specialitem{\tt \ttbackslash node%
                              \ttlbrace{\it Rechneradresse\/}\ttrbrace}
Bezeichnet die Adresse
des Rechners, auf dem die Benutzerkennung eingerichtet ist.
Voreingestellt
ist \verb+\node{@uni-muenster.de}+.
\begin{quotation}\noindent
\parbox[t]{70mm}%
{\ttbackslash userid\ttlbrace martina.mustermann@\ttrbrace\texttt{\ ==>\ \ }\\%
 \ttbackslash node\ttlbrace uni-muenster.de\ttrbrace}
\parbox[t]{40mm}%
{\scriptsize martina.mustermann@ \\ uni-muenster.de}
\end{quotation}

\specialitem{\tt \ttbackslash fax\ttlbrace{\it Nummer\/}\ttrbrace} %
Die Fax-Nummer wird unter der Telefonnummer eingefügt.
\begin{quotation}\noindent                                         %
\verb+\fax{(02\,51) 83-20\,90}    ==>  +%                          %
{\scriptsize Fax (02\,51) 83-20\,90}%                          %
\end{quotation}

\specialitem{\tt \ttbackslash datum\ttlbrace{\it Datum\/}\ttrbrace}
Das Datum kann in beliebiger Form eingegeben werden.
Ist diese Markierung nicht vorhanden so wird das Datum des
\TeX-Aufrufs eingesetzt.
\begin{quotation}\noindent
\verb+\datum{27.9.1985}           ==>  +27.9.1985\\
\verb+\datum{\today}              ==>  +27. September 1985
\end{quotation}

\specialitem{\tt \ttbackslash adresse%
                      \ttlbrace{\it Empfängeradresse\/}\ttrbrace}
Die Anschrift des Empfängers wird in das Brief\-fenster gesetzt.
\begin{quotation}\noindent
\begin{verbatim}
\adresse{Herrn Hans H. Haarmann\\
         Wasserweg 13\\
         12345 Beispielhausen}
\end{verbatim}
\adresse{Herrn Hans H. Haarmann\\
             Wasserweg 13\\
             12345 Beispielhausen}
\makeatletter
\verb+==>  +\vtop {\hsize .5\linewidth
                   \hbox{}\vspace*{-1cm}
                   \hbox{\schreibe@fensteradresse}}
\makeatother
\end{quotation}

\vspace*{-1cm}
\specialitem{\tt \ttbackslash nachrichtlich%
                      \ttlbrace{\it Empfängeradressen\/}\ttrbrace%
            }
Mit dieser Markierung lassen sich zusätzlich zur Empfängeradresse
weitere Adressen im Briefkopf erzeugen.
Die einzelnen Adressen werden jeweils durch eine Markierung der Form
\verb+\adresse{+{\it Adresse\/}\verb+}+ erzeugt. Es dürfen so viele
Adressen angegeben werden, wie auf die erste Seite passen. Die
zusätzlichen Adressen beginnen unter dem Brieffenster. Es ist auch
möglich die Schriftgröße zu ändern oder normalen Text auszugeben,
etwa das Wort \frqq Nachrichtlich:\flqq.\vspace{-4mm}
\begin{quotation}\noindent
\begin{verbatim}
\adresse{Peter Maurer\\Postweg 3\\
              12345 Beispielhausen}
\nachrichtlich{\small Nachrichtlich:\\[4ex]
    \adresse{Firma Maier\\Postfach\\
             12627 Berlin}
    }
\end{verbatim}
\adresse{Peter Maurer\\Postweg 3\\
                      12345 Beispielhausen}
\nachrichtlich{\small Nachrichtlich:\\[4ex]
             \adresse{Firma Maier\\Postfach\\
                     12627 Berlin}
             }
\makeatletter
\verb+==>  +\vtop {\hsize .5\linewidth\vspace*{-10mm}
                   \hbox{}\hbox{\vspace*{-10mm}\schreibe@fensteradresse}
                          \hbox{\schreibe@nachrichtlich}}
\makeatother
\vspace*{-1cm}
\end{quotation}


\specialitem{\tt \ttbackslash az\ttlbrace{\it Aktenzeichen\/}\ttrbrace}
Die Aktenzeichennotiz wird direkt unter das Datum gesetzt.
\begin{quotation}\noindent
\verb+\az{80/76}                  ==>  +Az.: 80/76
\end{quotation}

\specialitem{\parbox{6cm}
                    {\tt \ttbackslash bezugszeichen%
                           \ttlbrace{\it Textzeile\/}\ttrbrace\\[2mm]
                     \tt \ttbackslash betr%
                           [\ttbackslash skip\ttlbrace{\it n\/}\ttrbrace]%
                           \ttlbrace{\it Textzeile\/}\ttrbrace\\[2mm]
                     \tt \ttbackslash bezug
                           [\ttbackslash skip\ttlbrace{\it n\/}\ttrbrace]%
                           \ttlbrace{\it Textzeile\/}\ttrbrace%
                    }%
            }
Hiermit wird der Betreff und der Bezug des Briefes markiert.
\begin{quotation}
\makeatletter
\betr{Neuer Brief-Style}
\bezug{Ihre Anregung \ldots}
\noindent
\verb+\betr{Neuer Brief-Style}        ==>  +\\
\verb+\bezug{Ihre Anregung \ldots}         +
\parbox[t]{.3\linewidth}
                {\vspace*{-2.7\normalskip}\@betrtext\par
                 \vspace{\@bezugskip\normalskip}
                 \noindent\@bezugtext}\vspace{.5cm}
\makeatother
\end{quotation}

{\it Textzeile\/} ist ein beliebiger Text. Der optionale Stilparameter
\verb+\skip+ gibt an, wieviele Leerzeilen
vor Bezug oder Betreff eingefügt werden sollen. Voreingestellt
ist jeweils der Wert 2.
\begin{quotation}
\makeatletter
\bezugszeichen{Ihr Zeichen: E1 567}
\betr{Neuer Brief-Style}
\bezug[\skip{1}]{Ihre Anregung \ldots}
\noindent
\verb+\bezugszeichen{Ihr Zeichen:     ==>  +\\
\rule{0mm}{0mm}\verb+                   E1 567}           +\\
\verb+\betr{Neuer Brief-Style}             +\\
\verb+\bezug[\skip{1}]                     +\\
\rule{0mm}{0mm}\verb+      {Ihre Anregung \ldots}         +
\parbox[t]{.3\linewidth}
                 {\vspace*{-7.9\normalskip}\@bezugszeichen\par
                  \vspace{\@betrskip\normalskip}
                  \@betrtext\par
                  \vspace{\@bezugskip\normalskip}
                 \noindent\@bezugtext}
\makeatother
\end{quotation}

\specialitem{\tt \ttbackslash anrede%
                       [\ttbackslash skip\ttlbrace{\it n\/}\ttrbrace]%
                       \ttlbrace{\it Textzeile\/}\ttrbrace%
            }
{\it Textzeile\/} wird als Anrede gedruckt.
Der optionale Stilparameter \verb+\skip+ gibt an, wieviele Leerzeilen
vor der Anrede
freigelassen werden sollen. Voreingestellt ist der Wert 2.

\specialitem{\tt \ttbackslash grussformel\ttlbrace{\it Textzeile\/}\ttrbrace}
Mit dieser Markierung wird die Grußformel definiert, die am Ende des
Briefes gedruckt werden soll. Voreingestellt ist: \frqq Mit freundlichen
Grüßen\flqq.

\specialitem{\tt \ttbackslash text%
                      [\ttbackslash skip\ttlbrace{\it n\/}\ttrbrace]}
Diese Markierung beendet den Briefvorspann. Der gesamte Briefkopf bis
zur Anrede wird hiermit abgeschlossen. Nach dieser Markierung
beginnt der Rumpf mit dem Text des Briefes.
Der optionale Stilparameter \verb+\skip+ gibt an, wieviele
Leerzeilen vor dem Text
freigelassen werden sollen. Der Wert 1 ist voreingestellt.
%\pagebreak

\specialitem{\tt \ttbackslash schluss}
Diese Markierung bewirkt, daß der hiernach folgende Teil
des Briefes zusammenhängend
gedruckt wird. Er wird also entweder auf der laufenden Seite oder --
falls er nicht mehr auf diese Seite paßt -- auf die nächste Seite
gedruckt.

\specialitem{\tt \ttbackslash gruss%
                        [\ttbackslash skip\ttlbrace{\it n\/}\ttrbrace%
                         \ttbackslash ia%
                         \ttbackslash iv]%
                        \ttlbrace{\it Unterzeichner\/}\ttrbrace%
            }
Mit dieser Markierung wird ein Gruß unter den Brief geschrieben.
{\it Unterzeichner\/} wird als Name unter den Gruß
geschrieben. Wurde kein Name, d.h. \verb+{}+, angegeben, dann wird der
Name eingesetzt, der unter \verb+\absendername+ im Vorspann des Briefes
angegeben wurde. Mit \verb+{\ }+ kann die Ausgabe eines Namens
unterdrückt werden. Als optionale Argumente können angegeben werden:

\begin{description}[\setlabelsize{24mm}]
\item[{\tt \string\skip\ttlbrace{\it n\/}\ttrbrace}]Dieser Stilparameter
gibt an, wieviele
Leerzeilen zwischen dem Text und dem Gruß freigelassen werden sollen.
Voreingestellt ist der Wert 2.
\item[{\tt \string\ia}]Vor dem Namen des Grußes wird das Kürzel
\frqq i.A.\flqq\ eingefügt.
\item[{\tt \string\iv}]Vor dem Namen des Grußes wird das Kürzel
\frqq i.V.\flqq\ eingefügt.
\end{description}

Beispiel:
\begin{quotation}\noindent
\verb+\gruss[\ia]{H. Kaiser}      ==> +%
\parbox[t]{.4\linewidth}
{\textwidth=\linewidth\gruss[\ia\skip{-1}]{H. Kaiser}}
\end{quotation}

\specialitem{\tt \ttbackslash anlage%
                        [\ttbackslash skip\ttlbrace{\it n\/}\ttrbrace%
                         \ttbackslash wort\ttlbrace{\it Text\/}\ttrbrace%
                         \ttbackslash textrechts%
                         \ttbackslash textunten]%
                        \ttlbrace{\it Anlagenliste\/}\ttrbrace%
            }
Mit dieser Markierung kann auf eventuell vorhandene Anlagen
hingewiesen werden.
Der Stilparameter \verb+\skip+ gibt an, wieviele Leerzeilen
zwischen dem Gruß und dem Anlagenverzeichnis eingefügt werden sollen.
Voreingestellt ist der Wert 2.
Über den Stilparameter \verb+\wort+ kann festgelegt werden, welcher
{\it Text\/} vor oder über die {\it Anlagenliste\/} gesetzt werden
soll. Voreingestellt ist \verb+\wort{Anlage:}+.
Mit \verb+\textrechts+ oder
\verb+\textunten+ kann angegeben werden, ob die {\it Anlagenliste\/}
rechts neben den {\it Text\/} oder darunter gesetzt werden soll.
\verb+\textrechts+ ist dabei voreingestellt.
Um die Anlagen in der {\it Anlagenliste\/} untereinander ausgeben zu
lassen, sollten sie durch die Markierung \verb+\\+ voneinander getrennt
werden.
\begin{quotation}\noindent
\begin{verbatim}
\anlage[\wort{Anlagen:}\textunten]{Diskette\\Liste}
\end{verbatim}

\rule{0mm}{0mm}\verb+                         ==>  +%
\parbox[t]{.3\linewidth}
     {\anlage[\wort{Anlagen:}\textunten\skip{-1}]%
                      {Diskette\\Liste}}
\end{quotation}


\specialitem{\tt \ttbackslash ps%
                        [\ttbackslash skip\ttlbrace{\it n\/}\ttrbrace%
                         \ttbackslash wort\ttlbrace{\it Text\/}\ttrbrace%
                         \ttbackslash textrechts%
                         \ttbackslash textunten]%
                         \ttlbrace{\it Nachsatz\/}\ttrbrace%
            }
Mit dieser Markierung kann noch ein Nachsatz an den
fertigen Brief angehängt werden.
Der Stilparameter \verb+\skip+ gibt an, wieviele Leerzeilen
vor dem Postscriptum eingefügt werden sollen.
Voreingestellt ist der Wert 2.
Über den Stilparameter \verb+\wort+ kann festgelegt werden, welcher
{\it Text\/} vor oder über den {\it Nachsatz\/} gesetzt werden
soll. Voreingestellt ist \verb+\wort{PS:}+.
Mit \verb+\textrechts+ oder
\verb+\textunten+ kann angegeben werden, ob der {\it Nachsatz\/} rechts
neben den {\it Text\/} oder darunter gesetzt werden soll.
\verb+\textrechts+ ist dabei voreingestellt.
\begin{quotation}\noindent
\verb+\ps{Leider habe ich ...}  ==>  +PS: Leider habe ich \ldots
\end{quotation}

\specialitem{\tt \ttbackslash end\ttlbrace{\it Brief\/}\ttrbrace}
Markierung des Briefendes.
\end{description}
\vspace{2cm}

{\bf Zusätzliche Parameter}

\begin{description}[\setlabelsize{24mm}]

\specialitem{\tt \ttbackslash kopien\ttlbrace {\it
                                             Anzahl\/}\ttrbrace%
                           [{\it Ablagevermerke\/}]}%
Der optionale Parameter {\it Ablagevermerke\/} gibt an, was als
Ablagevermerk auf der ersten
Seiten einer jeden Kopien stehen soll. Pro Kopie steht hier eine
Ablagevermerkmarkierung, die über die Endziffer in
\verb+\ablagevermerk+{\it Nr\/} der entsprechenden Kopie zugeordnet
wird. Wurde die Anzahl der Kopien auf 3 gesetzt so stehen also
\verb+\ablagevermerk1{+{\it Text\/}\verb+}+,
\verb+\ablagevermerk2{+{\it Text\/}\verb+}+ und
\verb+\ablagevermerk3{+{\it Text\/}\verb+}+ zur Verfügung.
{\it Text\/} bezeichnet dabei den auf die Kopie zu schreibenden Text.
\verb+\ablagevermerk1{1. Kopie}+
bedeutet daher, daß auf der ersten Kopie oben
rechts \frqq 1. Kopie\flqq\
steht. Soll auf z.B. der zweiten Kopie das Aktenzeichen aus dem Brief
stehen,
kann dies durch \verb+\ablagevermerk2{\schreibeaz}+ erreicht werden.
Die
Voreinstellung für Kopien kann mit der Markierung
\verb+\kopien@voreinstellung+ (s.u.) vorgenommen werden. Standardmäßig
erscheint auf den ersten drei Kopien \frqq 1. Kopie\flqq,
\frqq 2. Kopie\flqq, \frqq 3. Kopie\flqq.

\specialitem{\tt \ttbackslash kopien@voreinstellung\ttlbrace {\it
                                             Anzahl\/}\ttrbrace%
                           [{\it Ablagevermerke\/}]}%
Diese Markierung hat die gleiche Form wie \verb+\kopien+. Die hiermit
definierten Ablagevermerke der einzelnen Kopien erscheinen, solange
nicht durch \verb+\kopien+ für die jeweilige Kopie explizit ein anderer
Ablagevermerk angegeben wird. Die Voreinstellung kann nur in der {\it
Briefkopfdatei\/} oder in der {\it Bearbeiterdatei\/} vorgenommen werden
(wegen des \verb+@+-Zeichens) und ist für dauerhafte Änderungen
vorgesehen.

\end{description}



\newpage
{\bf Beispielbrief:}
\vspace{6mm}


\begin{verbatim}
% Dies ist BEISP1.TEX vom 06.05.2008
% Beispiel zum Brief mit Uni-Briefkopf
\documentclass{wwubrief}
\usepackage{ngerman}
\usepackage[utf8]{inputenc}
\begin{document}
\begin{brief}{kopf_name}
%
%       ----- Beginn Brief-Vorspann -----
%
\kopien{1}[\ablagevermerk1{Die erste Kopie}]
\adresse{An alle \\
         Mitarbeiter des Instituts}
\absendername{Martina Mustermann}
\bearbeitertext{Bearbeiterin}
\telefon{12345}
\fax{12346}
\userid{martina.mustermann@\\}
\node{uni-muenster.de}
\betr{Neuer \TeX-Brief}
\bezug{Ihre Anregung \ldots}
\anrede{Liebe Kolleginnen und Kollegen,}
%
%       ----- Ende Brief-Vorspann -----
%
\text
%
%       ----- Beginn Brief-Rumpf -----
%
Hier steht der Brieftext, der beliebig formatiert werden kann.
\schluss
Zu diesem Zweck sind die normalen \TeX-Markierungen zu verwenden.
\gruss[\ia]{}
\anlage[\wort{\bf Anlagen:}]{Dokumentation\\
                             CD}
\ps[\skip{4}\textunten\wort{\bf PS:}]
  {Dies ist der formatierte Brief zu {\tt BEISP1.TEX}}
%
%       ----- Ende Brief-Rumpf -----
%
\end{brief}
\end{document}
\end{verbatim}


\newpage
{\bf Beispielbrief mit einer \textit{Bearbeiterdatei}:}
\vspace{6mm}


\begin{verbatim}
% Dies ist BEISP2.TEX vom 06.05.2008
% Beispiel zum Brief mit Uni-Briefkopf
\documentclass{wwubrief}
\usepackage{ngerman}
\usepackage[utf8]{inputenc}
\begin{document}
\begin{brief}[bearbeiter]{kopf_logo}
%
%       ----- Beginn Brief-Vorspann -----
%
\adresse{An alle \\
         Mitarbeiter des Instituts}
\betr{Neuer \TeX-Brief}
\bezug{Ihre Anregung \ldots}
\anrede{Liebe Kolleginnen und Kollegen,}
%
%       ----- Ende Brief-Vorspann -----
%
\text
%
%       ----- Beginn Brief-Rumpf -----
%
Hier steht der Brieftext, der beliebig formatiert werden kann.
\schluss
Zu diesem Zweck sind die normalen \TeX-Markierungen zu verwenden.
\gruss[\ia]{}
\anlage[\wort{\bf Anlagen:}]{Dokumentation\\
                             CD}
\ps[\skip{4}\textunten\wort{\bf PS:}]
  {Dies ist der formatierte Brief zu {\tt BEISP2.TEX}}
%
%       ----- Ende Brief-Rumpf -----
%
\end{brief}
\end{document}
\end{verbatim}

Die {\it Bearbeiterdatei\/} sieht dabei folgendermaßen aus:

\begin{verbatim}
% Dies ist BEARBEITER.STY vom 06.05.2008
\absendername{Martina Mustermann}
\bearbeitertext{Bearbeiterin}
\telefon{12345}
\fax{12346}
\userid{martina.mustermann@\\}
\node{uni-muenster.de}
\end{verbatim}

%--------------------------------------------------------
\newpage

\begin{brief}{kopf_name}
%
%       ----- Beginn Brief-Vorspann -----
%
\kopien{1}[\ablagevermerk1{Die erste Kopie}]
\adresse{An alle \\
         Mitarbeiter des Instituts}
\absendername{Martina Mustermann}
\bearbeitertext{Bearbeiterin}
\telefon{12345}
\fax{12346}
\userid{martina.mustermann@\\}
\node{uni-muenster.de}
\betr{Neuer \TeX-Brief}
\bezug{Ihre Anregung \ldots}
\anrede{Liebe Kolleginnen und Kollegen,}
%
%       ----- Ende Brief-Vorspann -----
%
\text
%
%       ----- Beginn Brief-Rumpf -----
%
Hier steht der Brieftext, der beliebig formatiert werden kann.
\schluss
Zu diesem Zweck sind die normalen \TeX-Markierungen zu verwenden.
\gruss[\ia]{}
\anlage[\wort{\bf Anlagen:}]{Dokumentation\\
                             CD}
\ps[\skip{4}\textunten\wort{\bf PS:}]
  {Dies ist der formatierte Brief zu {\tt BEISP1.TEX}}
%
%       ----- Ende Brief-Rumpf -----
%
\end{brief}

\newpage

\begin{brief}[bearbeiter]{kopf_logo}
%
%       ----- Beginn Brief-Vorspann -----
%
\adresse{An alle \\
         Mitarbeiter des Instituts}
\betr{Neuer \TeX-Brief}
\bezug{Ihre Anregung \ldots}
\anrede{Liebe Kolleginnen und Kollegen,}
%
%       ----- Ende Brief-Vorspann -----
%
\text
%
%       ----- Beginn Brief-Rumpf -----
%
Hier steht der Brieftext, der beliebig formatiert werden kann.
\schluss
Zu diesem Zweck sind die normalen \TeX-Markierungen zu verwenden.
\gruss[\ia]{}
\anlage[\wort{\bf Anlagen:}]{Dokumentation\\
                             CD}
\ps[\skip{4}\textunten\wort{\bf PS:}]
  {Dies ist der formatierte Brief zu {\tt BEISP2.TEX}}
%
%       ----- Ende Brief-Rumpf -----
%
\end{brief}

% -------------------------------------------------------------
\newpage

{\bf Beispiel für ein Rückantwortschreiben:}
\vspace{6mm}

\begin{verbatim}
% Dies ist RUECKANT.TEX vom 1.2.90 Ar
\documentclass{wwubrief}
\usepackage{ngerman}
\usepackage[utf8]{inputenc}
\begin{document}
\begin{brief}{rueckant}

\adresse{An das\\
         Institut XYZ\\
         an der WWU\\
         Herrn Dr. D. Veranstalter\\
         48149 Münster}

\text
Rückantwort zur XYZ-Veranstaltung am 11.11.11 zum Thema

{\bf Rückantwortschreiben:}

\begin{itemize}
\item Einleitung
\item Vortag von D. Veranstalter
\end{itemize}

\vspace{2cm}
Sehr geehrter Herr Dr. Veranstalter

an der Veranstaltung

\fbox{\phantom{M}}\ \ \ nehme ich teil

\fbox{\phantom{M}}\ \ \ nehme ich nicht teil

\gruss{(Unterschrift)}
\end{brief}
\end{document}
\end{verbatim}
\pagebreak


\begin{brief}{rueckant}

\adresse{An das\\
         Institut XYZ\\
         an der WWU\\
         Herrn Dr. D. Veranstalter\\
         48149 Münster}

\text
Rückantwort zur XYZ-Veranstaltung am 11.11.11 zum Thema

{\bf Rückantwortschreiben:}

\begin{itemize}
\item Einleitung
\item Vortag von D. Veranstalter
\end{itemize}

\vspace{2cm}
Sehr geehrter Herr Dr. Veranstalter

an der Veranstaltung

\fbox{\phantom{M}}\ \ \ nehme ich teil

\fbox{\phantom{M}}\ \ \ nehme ich nicht teil

\gruss{(Unterschrift)}
\end{brief}

\end{document}
